
\chapter{Introduction}
\label{ch:Introduction}



\section{Motivation}
\label{sec:Introduction:Motivation}

\section{Problem Statement}
\label{sec:Introduction:ProblemStatement}

\section{Challenges}
\label{sec:Introduction:Challenges}



\section{Stichpunkte}

\subsection{Approaches}
* \cite{ObjectAwareAmiri}: Object aware,identification from business processes using structural dependency and data object dependency\\
* \cite{ServiceCutter}: Service Cutter, service decomposition based on 16 coupling criteria extracted from literature and industry experience
\cite{interfaceAnalysisBaresi}: interface Analysis,  "automated process for identifying candidate ms my means of lightweight, domain.aghnostic semantic analysis of the concepts in input sepecification with regard ti a reference vocabulary" \\
\cite{FunctionalDecompositionHeinrich} : Functional decomp, systematic approach to identify ms in design phase by identifying relationships between required sytem operations and state variable that ops read/write, visualized in Graph and the clustering\\
\cite{DataflowDrivenChen}: Dataflow-driven, Top down dataflow-driven decomposition algorithm, purified dataflow-driven mechanism


\subsection{General Annotations}
* Several companies recentyl migrated to MS like Netflix, Amaozn, Uber
* not many approaches and identified approaches are not mature\\
* many use clustering \\
* based on code inspection, models... various approaches
* use chapter "Supporting Definitions"\\
* Fowler (widely adopted definition): an approach for developing a single application as a suite of small services, each running in its own process and communicationg with lightweight mechanisms, often a RESTful API. \\
* \cite{interfaceAnalysisBaresi}: nmot fully migrate application to ms, preserve monolithic structure and only replicate certain components... \\
Decomposition Approaches by (according to \cite{FunctionalDecompositionHeinrich}): \\
*Usecases(service that is responsible for particular action) 
*verbs (service that is responsible for particular action)
*nouns/resources (service responsible for all operations on entities/resources of a given type)
* business capability: Something that a business does in order to generate valu
* domain-driven design sub-domain (business splitted in different sub-domains) --> Seems to be most common approach


\subsection{Object-aware Identification}
\cite{ObjectAwareAmiri} \\
*Microservices:\\
*MS architectural style inspired by service-oriented computing \\
*cohesive and loosely coupled components, implement business capabilities (each service one capability)\\
* gain popularity in academia and inudtrsy\\
*overcomeshortcomings of centralized, monolithic architectures\\
* single app as suite off small services\\
* communication via leigthweight  mechanisms\\
* 3 principles: \\
* i)Bounded contexts(focused on business capabilities, related functionality implemented in one capability and implemented in one service)\\
* ii)  Size(if servize too large, split up, maintaining focus on providing one business capability only in one service)\\
* iii) Independence( loose coupling and high cohesion)\\
* --> encourages lossely... as each ms operates mostly independent of others \\ 
* independet collection of highly inter-related activities \\

Challenges:\\
* decompose system in cohesive lossely couped, fine-grained MS \\
* done intuitively by experience of system archjitects/Domain experts\\
* functionalities highly interconnected\\
* Problem well known \\

\subsection{Service Cutter}
\cite{ServiceCutter}

Microservice:\\
*important to split distributed systems in lc/hc units\\
*autonomous, network-accessible services.\\
* functional decomposition not new (1972 D.L. Parnas " Ont the Criteria to be used in Decomposing Systems into Modules)\\
* Allow developers to chose most appropriate technology for each particular business problem\\
* MS is "contemporary incarnation of SOA" combined with modern sw engineering practices like continous/idependetnt deployment\\
* decomposition og monolith in services not fully understood ("very much of an art")\\

Motivation: \\
* SW became more complex, sw engineers  started to distribute modules/functionality over network (Web Services)\\
* SOA tackled design challenge of distributed systems\\
* requirements-driven, repeatable and scalable service decomposition method (supported and partially automated by tools) does not exist\\
* use DDD: services derived from bounded contexts are aligned to domain model boundaries and team organization structures \\
* service accessesd remotely through invocation interface\\

Challenges:\\
* loosley coupled and highly cohesive services crucial for maintainability/scalability of software \\
* Not only DDD but also stakeholder requirements: Architecturally significant requirements (sw quality attributes) \\
* Non-functional requirements of sw is key aspect of analysis and design\\


\subsection{Microservice Identification Interface Analysis}
\cite{interfaceAnalysisBaresi}

General: \\
* MS evolution of SOA but differences: service reuse less; instead of reusing existing ms for new tasks they should be small enough to rapidly implement new one that can coexists, evolve, replace the previous one\\

Microservice:\\
* ms style: suite of small, autonomous, conversational services \\
* contrary to monoliths: independent deployability(scalability, using different technology stacks) \\
* boundaries between software module in tradiditonal services (SOA) often too flexible --> "big balls of mud" --> MS make boundaries physical\\
* partition ease system maintenance  and defines hgow system is able to evolve and scale \\



Motivation:\\
*overcome shortcomings of centralized monolithic architecture (deployed as a big chunk)\\
* definition of granularity level and trade-off between soze and number of MS is still blurred\\

Challenges: \\
* finding right granularity/cohesiveness (when starting or transforming project)\\
* identification extraordinary affects how system will be able to evolve --> define proper services is key challenge\\

\subsection{Functional Decompositions}
\cite{FunctionalDecompositionHeinrich}\\

General:\\


Microservices: \\
* rising fast, many companies use it to structure their systems\\
* usually defined intuitively based on experience of designers\\
*advantages: scalability (thus resillience), enhanced performance\\
* inteact via messages using standard data formats and protocols, publish interfaces using well defined lightweight mechanisms such as REST\\
* each microservice has own domain model (data, logic, behaviour), related functionality combined in one business capability (= bounded context), one MS implements capability (more if too big)\\
*tackles complexity of large application by decomposition in small pieces (each in own bounded context)\\
* architecture enable traceability between requirements and system structure --> Only 1 ms needs to be\\ changed/redeployed (overcome shortcomings of monoliths: Deployment easy but everything deployed to a server making changes impact whole application --> redeploy whole app)\\
* two main advantages: functional decomposition, decentralised governance\\
 



Challenges: \\
* find approppriate partition of system --> architecture can significatnly affects performance of system (intra-service calls?) \\
* lack of systematic approaches \\
* hardly any guidelines on what is "a good size" of ms --> Differs from system to system according to research\\
* two concepts for DB\\
* i) Share nothing: Each ms has own db --> Higher speed, horizontal scalability, but price of data consistency (only evenutal consistency)\\
*ii)"microservice is not an island" --> share DB but price of less Independence\\


\subsection{Dataflow-Driven Approach}
\cite{DataflowDrivenChen} \\

General: \\
* Monolith Strength: simple to develop, test, deply, scale. But: Grow in size --> monstrous monolithic architecture, complex/incomprehensible code structure, hold back bug fixes, slow down development, obtsacle for CD\\
* particular approach for SOA but no consensus on relationship of ms and SOA \\
* Advantages  accepted in academia and industry\\
* Advantages: Netflix can deal with billion calls every day by streaming API in ms structure\\
* decomposition process represented by Y-Axix of scaling cube\\
* decomposition by business capability and by domain-.driven design sub domain (abstract patterns, require human involvement); by verb/use cases or nouns/resources (easier to realize automation as lonmg as criteria have been predefined)\\
* Service cutter lack objectivity: Scoring the edge of the graph\\
*SOA: Services coarse-grained (vs. fine grained), decomposition aims at selecting the optimal composed service from all possible service combinations regarding quality requirements --> Bottom Up (vs. MS decomposition is top-down partition, then bottom-up integration)\\

Microservices: \\
* multiple small-scale and independently deployable ms\\
* design time features: Coupling between ms \\
*runtime feature: scalability against changing workload\\
* three design time principles:\\
* i) Fine grain and focus: small, autonomous as main characteristics, each ms does "one thing well"\\
* ii) High cohesion/loosae coupling: ms implement relatively independent piece of business logic (hc), microservice should barely depend on other (lc) \\
*iii) Neutral devel. technology, services should be deployable as individual applications with own delivery pipeline \\


Challenges:\\
* ms not panacea, main issue is effectively decompose a monolithic application
* commonly manual decomposition process prevent practice for achieving those benefits



\subsection{Domain Engineering approach}
\cite{DomainEngineeringMunezero} \\

General:\\


Microservice:\\




Challenges: \\






