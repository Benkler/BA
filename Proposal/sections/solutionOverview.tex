\chapter{Solution Overview}
\label{ch:SolutionOverview}
\begin{itemize}
	\item Matrix-based approach: Starting from activities in business processes -> Each Activity is one node, pairwise connection between activities are edges, represented by weight
	
	\item Matrix so far: 
	\begin{itemize}
		\item Structural Dependency: Connection (direct or through one ore more gates) of two activities (high cohesion)
		\item Object Dependency: Read/Write Access to same objects (loosely coupled)

	\end{itemize} 

	\item Matrix to define data flows
	\item Security can be added if necessary (need data flow to determine security relevant data)
	\item Lübke et al. present algorithm to visualize use case sets as BPMN Processes
	\item No paper found to transform use cases in DFDs! How connect DFD and BPMN while keeping Matrix-Approach
\end{itemize}

\vspace{2cm}
\textbf{Questions:}
\begin{itemize}
	\item Extract business process from UC as part of thesis?
	\item Identify Objects from UC, or from existing OEM Model or what?
	\item Clustering algorithm (currently using Turbo MQ of "Bunch"-Software) --> Try other Algorithms?
	\item Why is adding the matrices useful? --> Empirical testing?
	\item How determine weights? (1,  0.25 etc...) --> What about different weighting for "more important" reads/writes? (Probably leads to need of user know-how)
	\item REST more expensive than DB-Call? --> Does not matter if read or write?
	\item Caching might explain weights --> Read probably cached? Cheaper that write
	\item Data flow has to be determined between activities (or approach cannot be applied)
	\item What about all the missing use cases in CoCoME (Create Supplier etc...)
	
\end{itemize}