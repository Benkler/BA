

\Abstract
Powered by the rise of cloud computing, agile development, DevOps and continuous deployment strategies, the microservice architectural pattern emerged as an alternative to monolithic software design. Microservices, as a suite of independent, highly cohesive and loosely coupled services, overcome the shortcoming of centralized monolithic architectures. Therefore, prominent companies recently migrated their monolithic legacy applications successfully to microservice-based architecture. They key challenge is to find an appropriate partition of legacy applications - namely \textit{microservice identification}. So far, the identification process is done intuitively based on the experience of system architects and software engineers - mainly by virtue of missing formal approaches and a lack of automated tool support. \\
However, when application grow in size and become progressively complex, it is quite demanding to decompose the system in appropriate microservices.
This thesis provides a formal identification model to tackle this challenge. The identification process is based on... // Hier jetzt noch kurz beschreiben was ich eigentlich machen will
We use 




