

\chapter{State of the Art}
\label{ch:StateOfTheArt}
This chapter outlines the current state of the art regarding microservice identification.  Sec. \ref{sec:StateOfTheArt:LiteratureReview} presents the search strategy and several existing approaches (Table \ref{tab:overviewLiterature}) that deal with the identification of microservices. Thereupon, the approaches are further explained and finally compared on the basis of several criteria.

\section{Literature Review}
\label{sec:StateOfTheArt:LiteratureReview}
The approaches mentioned in table \ref{tab:overviewLiterature} are the result of an extensive literature research which was conducted using the digital libraries IEEE \footnote{http://ieeexplore.ieee.org }, ACM \footnote{http://portal.acm.org} and SpringerLink \footnote{http://www.springerlink.com }. The web search enginge Google Scholar \footnote{http://scholar.google.com} provided further approaches and general information. \cite{FunctionalDecompositionHeinrich} was provided by the supervisor of this thesis and \cite{ServiceCutter} was cited by various approaches, including \cite{interfaceAnalysisBaresi}. The following search string was used:

\begin{centering}
{\itshape
   ["identify" OR "identification" OR "migrating" OR "monolith" OR "decomposition" OR "decompose monolith"
  	OR "decompose"] AND  "microservice"  \\
  	   OR \\  "microservice"  AND ["identification" OR "transformation" OR "refactor"]
}
 
   
\end{centering}

\noindent
Table \ref{tab:overviewLiterature} presents the 8 most promising approaches regarding the criteria mentioned in table //TODO!!!!!!. Now, a short introduction to each approach is given.




\begin{table}[h!]

\centering
     
	
	 \rowcolors{2}{gray!25}{white}
	\begin{tabularx}{\textwidth}{lXXlX}
		\rowcolor{gray!50}
		Link & Titel & Author   & Origin & Search String  \\
		
		\rowcolor{gray!50}
		& & (Year) & & \\
		
		\cite{ExtractionMazlami} & Extraction of Microservices from Monolithic Software Architectures  & G. Matzlami et. al. (2017) & Google Scholar&  {\itshape microservice identification }  \\
		
		
		\cite{ObjectAwareAmiri} & Object-Aware Identification of Microservice & M. J. Amiri (2018) & IEEE & \textit{identification microservices}\\\
		
		\cite{interfaceAnalysisBaresi} & Microservices Identification Through Interface Analysis & L. Baresi et. al. (2017)& SpringerLink & \textit{microservice identification}\\
		
		
		
		 
		 \cite{FunctionalDecompositionHeinrich}& Identifying Microservices Using Functional Decomposition & S. Tyszberowicz et. al. (2018) & \textit{provided} & \textit{n/a} \\
		 
		 \cite{DomainEngineeringMunezero} & Partitioning Microservices: A Domain Engineering Approach & I. J. Munezero et. al. (2018) & ACM & \textit{partition microservices}\\
		 
		 
		 \cite{DataflowDrivenChen} & From Monolith to Microservices: A Dataflow-Driven Approach & R.Chen et. al & IEEE & monolith microservice \\
		 
		\cite{HeuristicsAlwis} & Function-Splitting Heuristics for Discovery of Microservices in Enterprise Systems & A. De Alwis et. al. (2018 )& Google Scholar & identify microservices \\
		
	\cite{ServiceCutter} & 	Service Cutter: A Systematic Approach to Service Decomposition& M. Gysel et. al. (2016) & \cite{interfaceAnalysisBaresi} & \textit{n/a} \\
	
	\end{tabularx}
	\caption{List of authors and approaches}
	\label{tab:overviewLiterature}
\end{table}


\clearpage





\section{Approaches}


\noindent
\textbf{Extraction of Microservices from Monolithic Software Architectures   } \\
The approach presented in \cite{ExtractionMazlami} is a class based extraction model, that uses (meta-)information of a version control system \textit{(VCS)} such as Git\footnote{https://github.com/} to identify microservices. The approach is divided in two phases: The \textit{Construction Phase} and the \textit{Clustering Phase}.
Starting with a given code base, the approach uses three different coupling strategies and the information provided by the \textit{VCS} to transform the monolith into a weighted graph. Here, the nodes represent classes, and the edges have weights according to the chosen coupling strategy. In the second phase, a clustering algorithm determines possible microservices (each cluster is a microservice candidate). \\

\noindent
\textbf{Object-Aware Identification of Microservice  } \\
\cite{ObjectAwareAmiri} identifies microservices from business processes, using the widely known \textit{Business Process and Model Notation (BPMN)}. The activities in a business process represent functionality in the system. They also perform read and write operations on data objects. The goal is to identify microservices by using the structural and data object dependencies of the business processes. Therefore, the author proposes a clustering technique


\cite{ObjectAwareAmiri} uses clustering based on structural dependency and data object dependency. The system is modelled 





The approach models a system as a set of business processes using the well known graphical representation 
\noindent
\textbf{Microservices Identification Through Interface Analysis   } \\


\noindent
\textbf{Identifying Microservices Using Functional Decomposition  } \\


\noindent
\textbf{Partitioning Microservices: A Domain Engineering Approach } \\


\noindent
\textbf{From Monolith to Microservices: A Dataflow-Driven Approach } \\

\noindent
\textbf{Function-Splitting Heuristics for Discovery of Microservices in Enterprise Systems  } \\


\noindent
\textbf{Service Cutter: A Systematic Approach to Service Decomposition  } \\



