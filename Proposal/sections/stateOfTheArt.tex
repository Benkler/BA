

\chapter{State of the Art}
\label{ch:StateOfTheArt}

\section{Literature Review}
\label{sec:StateOfTheArt:LiteratureReview}

\begin{table}
\centering
     
	
	 \rowcolors{2}{gray!25}{white}
	\begin{tabularx}{\linewidth}{lXXlX}
		\rowcolor{gray!50}
		Link & Titel & Author   & Origin & Search String  \\
		
		\rowcolor{gray!50}
		& & (Year) & & \\
		
		\cite{ExtractionMazlami} & Extraction of Microservices from Monolithic Software Architectures  & G. Matzlami et. al. (2017) & Google Scholar&  {\itshape microservice identification }  \\
		
		
		\cite{ObjectAwareAmiri} & Object-Aware Identification of Microservice & M. J. Amiri (2018) & IEEE & \textit{identification microservices}\\\
		
		\cite{interfaceAnalysisBaresi} & Microservices Identification Through Interface Analysis & L. Baresi et. al. (2017)& google scholar & \textit{microservice identification}\\
		
		
		
		 
		 \cite{FunctionalDecompositionHeinrich}& Identifying Microservices Using Functional Decomposition & S. Tyszberowicz et. al. (2018) & \textit{provided} & \textit{n/a} \\
		 
		 \cite{DomainEngineeringMunezero} & Partitioning Microservices: A Domain Engineering Approach & I. J. Munezero et. al. (2018) & IEEE & \textit{identify microservices}\\
		 
		 
		 \cite{DataflowDrivenChen} & From Monolith to Microservices: A Dataflow-Driven Approach & R.Chen et. al & IEEE & monolith to microservice \\
		 
		\cite{HeuristicsAlwis} & Function-Splitting Heuristics for Discovery of Microservices in Enterprise Systems & A. De Alwis et. al. (2018 )& Google Scholar & identify microservices \\
		
	\cite{ServiceCutter} & 	Service Cutter: A Systematic Approach to Service Decomposition& M. Gysel et. al. (2016) & \cite{interfaceAnalysisBaresi} & \textit{n/a} \\
	\end{tabularx}
	
\end{table}




\section{Comparison and applicability of the approaches}
\label{sec:StateOfTheArt:ComparisonAndApplicability}