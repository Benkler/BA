\chapter{CoCoME}
\label{ch:CoCoME}
The \textit{Common Component Modelling Example (CoCoME)} is a common case study on software architecture modelling \cite{CoCoMEOld}\cite{CoCoMETechnical}. In this thesis, it is used to demonstrate and validate the presented approach. Sec.\ref{sec:CoCoME:Introduction} provides a short introduction of the demonstrator, followed by a presentation of its system specifications.


\section{Introduction to CoCoME}
\label{sec:CoCoME:Introduction}
CoCoME represents a trading system as it can be found in a supermarket chain. The main task is handling and processing sales at a single store of the chain. Therefore, customers can pick goods and place them on the \textit{Cash Desk} whose main component is a \textit{Cash Desk PC}. Several other components like \textit{Bar Code Scanner}, \textit{Light Display}, \textit{Printer}, \textit{Card Reader} and \textit{Cash Box} are wired by the  \textit{Cash Desk PC}. \\
Multiple  \textit{Cash Desks} of a single store form a  \textit{Cash Desk Line}, which is connected to the  \textit{store server}. A set of stores in the CoCoME chain is organized as an enterprise where each store is connected to a single enterprise server. \\
More detailed description of the CoCoME system can be found in \cite{CoCoMEOld}\cite{CoCoMETechnical}. The next section provides information about the system requirements specifications in form of use cases.
 


\section{System specifications}
\label{sec:CoCoME:systemSpecifications}
