\chapter{Evaluation Planning}
\label{ch:EvalutationPlanning}
In this chapter, the evaluation strategy is outlined. The \textit{Research Questions} mentioned in \ref{sec:Introduction:ResearchQuestions} do not only induce the elaboration of an approach to identify microservices, but also raise the question about the quality of the formulated approach (\textbf{RQ3}). It is therefore necessary to apply it and compare the results with an existing approach. In the following, the procedure is further enlightened:

\section{Applicability to Case Study}
\label{sec:EvaluationPlanning:ApplicabilityToCoCoME}
The \textit{Common Component Modelling Example}, or in short, \textit{CoCoME} acts as the demonstrator system. It is  introduced in Chapter \ref{ch:CoCoME}. \textit{CoCoME} abstracts an enterprise system as it can be found in a supermarket chain. Having regard to the evaluation process, the first task is to define the requirements of the approach. Once they are delineated, the required information has to be extracted from the informal system specifications of the demonstrator, which are given in \cite{CoCoMEOld} and \cite{CoCoMETechnical}. Afterwards, the approach is ready to be applied to the system. The results are discussed by comparing them with the outcome presented in \cite{FunctionalDecompositionHeinrich} and a manual identification and implementation conducted by the author of this thesis.


\section{Comparison to Functional Decomposition Approach}
\textit{Identifying Microservices Using Functional Decomposition} \cite{FunctionalDecompositionHeinrich} is a systematic approach to find a appropriate partition of a system into microservices. This paper emerged as a result of the collaboration of the Academic Colge Tel-Aviv Yafo, the Karlsruhe Institute of Technology and the Southwest University China and uses CoCoME as demonstrator as well. Nevertheless, the approach has some disadvantages and limitations as depicted in Sec.\ref{sec:stateOfTheArt:comparison}. 
\\
However, one can presume that the proposed microservices in \cite{FunctionalDecompositionHeinrich} are good candidates, as their evaluation included three independent software projects that implemented CoCoME in a similar manner. \\
Accordingly, this thesis elaborates the similarities and differences between both decompositions in order to validate the quality of the new approach.
