\chapter{Conclusion}
\label{ch:Conclusion}
This thesis focused on finding a formal way to extract microservices using clustering on control flow and data flow. Chapter \ref{ch:Introduction} motivates the topic and poses the research questions. In Chapter \ref{ch:background}, the monolithic software architecture and the microservice architecture is introduced and shortly compared. Further, challenges and benefits of the microservice architecture are outlined. Also, this chapter introduces use cases and the \textit{Business Process and Model Notation (BPMN)}, which is a graph oriented language to describe business processes. Chapter \ref{ch:CoCoME} presents \textit{CoCoME}, the running example used to evaluate the approach. \textit{CoCoME} is a community case study and has recently been used in other projects to evaluate the results of microservice identification approaches. The current state of the art is introduced in Chapter \ref{ch:StateOfTheArt}, which provides the formal answer to \textbf{RQ1.1}: Several promising approaches and attempts to extract microservices are identified. Most of them require either a vast amount of initial input in form of diagrams, log files, graphs etc. or they require non-trivial user interaction during the extraction process. Also, some approaches cannot be applied to greenfield applications, as they need existing source code or at least a the change history of the software development process. However, a main objective was to identify microservice candidates without detailed know-how and manual effort which non of them can satisfy. \\
To answer \textbf{RQ1.2}, Chapter \ref{ch:Solution} introduces the actual approach to identify microservices based on business process models.  The approach is divided in several steps, where each step is explained in a distinct section. First, the BPMN models are specified. Since CoCoME's systems specifications are given in form of detailed use cases, a transformation method is introduced to transform use cases in BPMN models. In the following, strategies to extract the data flow and the control flow of the BPMN models are explained. Subsequently, a procedure is explained to visualize the data and control flow information as two separate weighted graphs that represent the dependencies between either data objects or activities. The next step introduces a clustering algorithm that uses a fitness function to determine the fitness of a selected partition. This algorithm is applied to the graphs in order to identify high cohesive and loosely coupled activity and data object clusters. In the last steps of the approach, four different methods to match the activity and data clusters are presented, where each combined cluster corresponds to a possible microservice candidate. \\
Chapter \ref{ch:SolutionApplication} applies the approach to the running example. Chapter \ref{ch:Evalutation} introduces a \textit{GQM Plan} to answer \textbf{RQ1.3} efficiently. This includes the introduction of a metric and the presentation of two reference sets. Further, the results of the previous chapter are compared to the reference sets using the metric.


\section{Outcomes}


\vspace{0.5cm}
\par
\begingroup

\noindent
\textbf{RQ1: How to identify microservices based on the system specifications? }

\vspace{0.3cm}
\noindent
\textbf{Outcome:} Test

\endgroup
\vspace{0.5cm}


\vspace{0.5cm}
\par
\begingroup

\noindent
\textbf{RQ1.1: Which is an appropriate strategy to decompose a system into microservices? }

\vspace{0.3cm}
\noindent
\textbf{Outcome:} Test

\endgroup
\vspace{0.5cm}

\vspace{0.5cm}
\par
\begingroup

\noindent
\textbf{RQ1.2: What formal approach can be constructed to identify possible microservices without detailed know-how and manual effort? }

\vspace{0.3cm}
\noindent
\textbf{Outcome:} Test

\endgroup
\vspace{0.5cm}

\vspace{0.5cm}
\par
\begingroup

\noindent
\textbf{RQ1.3: What is the accuracy of the approach? }

\vspace{0.3cm}
\noindent
\textbf{Outcome:} Test

\endgroup
\vspace{0.5cm}


\section{Discussion}







sehr schnell


Wollen gar nicht darübe reden, ob die Services gut sind im Sinne von Microservice gut


Accuracy:
The overall results are sophisticating: Apart from the functionality recall value, any other value exceeds 0.7. In other words, the approach identifies on average more than 70   

\section{Limitations and Future Work}
Granularität der Business Prozesse...


Matching Algo weiterführen



\section{Using Use Cases as Input}
- Das Transformieren mit Data Needs ist nicht komplett trivial --> Falls aber BPMN als Input bereits vorliegen, wäre es perfekt

\section{Control Flow Graph...}
- Bei direkt verbundenen ist es eindeutig. Bei Parallel Gateway auch, da beide gleich oft ausgeführt werden. Problematisch wird es bei XOR, da man nicht weiß, welche der wege mit welcher Wahrscheinlichkeit genommen werden



Rein das wir kein extra DFD wollen als inpout sonder einfachen input verarbeiten --> mehr testen und eventuell doch den DFD extra extrahieren


Sind denn alle zugriffe gleichwertig? Gbt es unterschiede


Das mit dem 4 matching ansatz könnte problem mit grob/feingranular lösen! --> Limitations