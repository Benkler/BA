\chapter{Conclusion}
\label{ch:Conclusion}
%TODO chapter beschreiben


\section{Outcomes}


\vspace{0.5cm}
\par
\begingroup

\noindent
\textbf{RQ1: How to identify microservices based on the system specifications? }

\vspace{0.3cm}
\noindent
\textbf{Outcome:} Test

\endgroup
\vspace{0.5cm}


\vspace{0.5cm}
\par
\begingroup

\noindent
\textbf{RQ1.1: Which is an appropriate strategy to decompose a system into microservices? }

\vspace{0.3cm}
\noindent
\textbf{Outcome:} Test

\endgroup
\vspace{0.5cm}

\vspace{0.5cm}
\par
\begingroup

\noindent
\textbf{RQ1.2: What formal approach can be constructed to identify possible microservices without detailed know-how and manual effort? }

\vspace{0.3cm}
\noindent
\textbf{Outcome:} Test

\endgroup
\vspace{0.5cm}

\vspace{0.5cm}
\par
\begingroup

\noindent
\textbf{RQ1.3: What is the accuracy of the approach? }

\vspace{0.3cm}
\noindent
\textbf{Outcome:} Test

\endgroup
\vspace{0.5cm}


\section{Discussion}







sehr schnell


Wollen gar nicht darübe reden, ob die Services gut sind im Sinne von Microservice gut


Accuracy:
The overall results are sophisticating: Apart from the functionality recall value, any other value exceeds 0.7. In other words, the approach identifies on average more than 70   

\section{Limitations and Future Work}
Granularität der Business Prozesse...


Matching Algo weiterführen



\section{Using Use Cases as Input}
- Das Transformieren mit Data Needs ist nicht komplett trivial --> Falls aber BPMN als Input bereits vorliegen, wäre es perfekt

\section{Control Flow Graph...}
- Bei direkt verbundenen ist es eindeutig. Bei Parallel Gateway auch, da beide gleich oft ausgeführt werden. Problematisch wird es bei XOR, da man nicht weiß, welche der wege mit welcher Wahrscheinlichkeit genommen werden



Rein das wir kein extra DFD wollen als inpout sonder einfachen input verarbeiten --> mehr testen und eventuell doch den DFD extra extrahieren


Sind denn alle zugriffe gleichwertig? Gbt es unterschiede


Das mit dem 4 matching ansatz könnte problem mit grob/feingranular lösen! --> Limitations