\chapter{Conclusion}
\label{ch:Conclusion}
In the following section the chapters are summarized. Further, the elaborated approach is discussed. The presentation of the limitations and the future work finish the thesis.


\section{Summary}
The aim of this thesis was finding a way to extract microservices using clustering on control flow and data flow. In Chapter \ref{ch:Introduction}, the topic is motivated and the research questions are posed. In Chapter \ref{ch:background}, the monolithic software architecture and the microservice architecture is introduced and shortly compared. Further, challenges and benefits of the microservice architecture are outlined. Also, this chapter introduces use cases and the \textit{Business Process Model and Notation (BPMN)}, which is a graph oriented language to describe business processes. In Chapter \ref{ch:CoCoME} \textit{CoCoME}, the running example used to evaluate the approach is presented. \textit{CoCoME} is a community case study and has recently been used in other projects to evaluate the results of microservice identification approaches. The current state of the art is introduced in Chapter \ref{ch:StateOfTheArt}: Several promising approaches and attempts to extract microservices are identified. Most of them require either a vast amount of initial input in form of diagrams, log files, graphs etc. or they require non-trivial user interaction during the extraction process. Also, some approaches cannot be applied to greenfield applications, as they need existing source code or at least the change history of the software development process. However, a main objective was to identify microservice candidates without detailed knowledge and manual effort which non of them can satisfy. \\

\noindent
Inspired by the current State of the Art, the following strategy to decompose a system into microservices is defined (\textbf{RQ1.1}): \\
The control flow and the data flow are used to identify two separate sets of clusters: Activity clusters based on control flow dependencies and data object clusters based on data flow dependencies. Afterwards, both sets of clusters are matched in order to generate highly cohesive and loosely coupled microservices. \\

\noindent
To answer \textbf{RQ1.2}, the actual approach to identify microservices based on business process models is introduced in chapter \ref{ch:Solution}. The approach is divided in several steps, each step is explained in a distinct section. First, the BPMN models are specified. Since CoCoME's systems specifications are given in form of detailed use cases, a transformation method is introduced to transform use cases in BPMN models. In the following, strategies to extract the data flow and the control flow of the BPMN models are explained. Subsequently, a procedure is explained to visualize the data and control flow information as two separate weighted graphs that represent the dependencies between either data objects or activities. The next step introduces a clustering algorithm that uses a fitness function to determine the fitness of a selected partition. This algorithm is applied to the graphs in order to identify highly cohesive and loosely coupled activity and data object clusters. In the last steps of the approach, four different methods to match the activity and data clusters are presented, where each combined cluster corresponds to a possible microservice candidate. \\
Still, the manual effort is not negligible as the approach is not yet implemented. However, no additional effort, knowledge or user-interaction is necessary, as soon as the BPMN models are specified. \\

\noindent
In chapter \ref{ch:SolutionApplication}, the approach is applied to the running example in order to generate a microservice decomposition. Afterwards, the approach is evaluated in chapter \ref{ch:Evalutation} by comparing the identified microservices with two reference sets, using a metric called \textit{Precision and Recall}. This provides information about the accuracy of the approach to answer \textbf{RQ1.3}. The evaluation came to the following result:\\
The overall accuracy of the approach is satisfying. No false microservice was identified when applying the approach to CoCoME. Only one of the identified microservices is more coarse-grained compared to the reference sets. In addition, most of the identified functionalities and data objects were allocated to the right service. \\







\section{Discussion}
Using use cases as input causes some trouble as the transformation into BPMN models with data needs is not trivial. It rather requires manual effort to detect synonymous tasks and data objects among all use cases to prevent a distorted result. This opposes the main purpose of the approach which is to reduce the required manual work and effort. However, it would be very difficult to create an approach that is capable of handling every form of input. Consequently, the first step, which is specifying the BPMN models, should rather be considered as prerequisite of the approach instead of being part of it. In case the system specifications are given as BPMN models with accompanying data reads and writes, the approach gets by on a minimum amount of required expertise and effort. \\
Section \ref{sec:Solution:CreateGraphControl} introduces the creation of a graph based on the control flow information. Regarding possible alternatives in the control flow (XOR gateways), it was decided not to assess different choices in terms of different weighting. Yet, it is debatable whether this does not distort the actual control flow dependencies and consequently the results. Since the goal was not to identify perfect and fixed microservices, but to deliver candidates with little additional knowledge, this point is rather neglectable. \\
Another debatable topic is the equal weighting of all edges between activities. One can argue that some activities are more relevant and executed more often. For instance, the \textit{Sale} process is probably performed more frequently than the creation a new product. Hence, activities covering the \textit{Sale} process need to be connected using edges with higher weights to raise the probability of ending in the same service. However, this would require domain experts to judge the relevance of each activity, leading to a much more complex and user-interactive process. \\ 
Probably the most controversial issue is the identification of data flow dependencies (cf. Sec.\ref{sec:Solution:CreateGraphData}). As BPMN 2.0 is not capable to model the data flow but rather the data needs of the activities, it is necessary to extract data flow dependencies between two data object based on heuristics. However, these depend very much on the granularity of the BPMN models. The value for the parameter \textit{n} needs to be adapted to the granularity, where fine-grained business processes demand for a higher value and vice versa. Additionally, the approach may not produce satisfying results if the granularity varies, that is if some business processes are much more fine or coarse grained than others. Consequently, it is perhaps necessary to think about additional input: The BPMN models are still used to extract the control flow information and to create activity clusters. In regard to data clusters, it might be necessary to use actual data flow diagrams that represent the exact data flow dependencies in order to create the data object related graph. Although this increases the necessary amount of input, it might improve the quality of the identified microservices. It is worth mentioning that the approach itself only needs to be adapted at two positions: i) \textit{Extract Data Flow} and ii) \textit{Create a weighted Graph using Data Flow}, both  adaptions would be trivial steps if the input is expanded by data flow diagrams. \\
The evaluation also demonstrated how close the result is compared to an actual implementation. However, it shows how important a complete system specification is: The use case specification of CoCoME does not mention tasks like \textit{Create Store} or \textit{Create Products}. Consequently, the result does not contain these either. Since the approach requires no further human intervention, the input must be very precise and detailed.\\
The overall results are satisfactory. The identified microservices, including their functionalities and data objects are mostly correctly identified. Compared to the manual process as presented in Sec.\ref{sec:Evaluation:ReferenceSets} which took days, the approach presented requires only a few hours. 







\section{Limitations and Future Work}
\label{sec:Conclusion:LimitsFutureWork}
Despite the overall satisfying results, the approach has its limitations. First of all, system requirements need to be transformed into BPMN 2.0 models with additional data needs. Even though the BPMN language is well known to describe business processes, the extension with the data objects is rather unusual. Hence, specifying those models might be a new task for the system designers.  \\
In addition, it needs to be verified whether the approach is capable to detect different granular microservices. For instance, small services like \textit{Order} and \textit{Product} are not identified by the approach, because it merges them to a more coarse-grained microservice which has approximately the size of the other identified services. In this case, further research on different clustering algorithms needs to be conducted. Also, the possibility to choose the degree of granularity when applying the approach would be desirable. The current algorithm is not capable of finding various decompositions of a graph that vary in size. \\
In addition, a conceptual algorithm to match cluster by using merging but also splitting was presented. This needs further attention as it might solve the issue of different granularity of microservices.  Above all, the currently used black box cluster matching process can lead to very large microservices, since only merging is used. Therefore, the conceptional white box clustering algorithm should be further investigated. \\
As already discussed in the previous section, the granularity of the BPMN models determines the choice of the parameter \textit{n}, which influences possible data object dependencies. 
Apart from the fact that a wrong choice of the parameter distorts the result, it is questionable to apply the approach to BPMN models of different granularity. In other words, the approach is able to handle either coarse-grained or fine-grained BPMN models, but not a mix of both. \\
The microservice candidates generated by the approach are not fix and ready to implement but should rather be used as a recommendation to inspire the system architect. Additionally, microservices need a clear and well-defined external interface to provide shared functionality and data access to other services. This is not yet covered by the approach, although the information is already provided implicitly. 



