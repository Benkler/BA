\chapter{Evaluation}
\label{ch:Evalutation}
The introduction chapter presents a \textit{Research Question} for the elaborated approach and its evaluation:


\vspace{0.5cm}
\par
\begingroup
\leftskip=1cm
\rightskip=1cm

\noindent
\textbf{RQ3: What is the accuracy of the approach? }

\endgroup
\vspace{0.5cm}

\noindent
To tackle this question, a \textit{Goal Quality Metrics Plan } (GQM) is introduced to specify the key aspects of the evaluation. In a word, the elaborated approach is used to identify a set of microservice candidates which is compared to two reference sets of microservices. With this in mind, the \textit{Precision and Recall} metric is used to determine the accuracy of the elaborated approach.




\section{Goal Quality and Metrics}
\label{sec:Evaluation:GQM}
The \textit{GQM Plan} defines the goal of the evaluation on a conceptual level. Further, questions are defined to achieve the specific goal. To answer the questions in a measurable way, metrics have to be defined that are associated with the questions. \\
In the following, the \textit{GQM Plan} is shortly described:

\begin{itemize}
	\item \textbf{G1:} Determine the accuracy of the approach
	\item \textbf{G1.Q1:} What is the \textit{Precision and Recall} of the identified microservices compared to the reference amount?
	\item \textbf{G1.Q1.M1:}  Precision and Recall
	\item WAS WAR HIER NOHCMAL MIT DEN ZYKLISCHEN ABHÄNGIGKEITEN
\end{itemize}


\section{Reference Amount}
To evaluate the approach, the identified set of microservices (cf. Sec.\ref{sec:Evalutation:Results}) is compared to two alternative decompositions of the case study: First, a decomposition proposed in the paper \textit{Identifying Microservices Using Functional Decomposition} \cite{FunctionalDecompositionHeinrich} and second, a set of microservices which we manually identified. \\



\subsection{Reference Set 1: Functional Decomposition Approach}
\textit{Identifying Microservices Using Functional Decomposition} \cite{FunctionalDecompositionHeinrich} is a systematic approach to find a appropriate partition of a system into microservices. This paper emerged as a result of the collaboration of the Academic College Tel-Aviv Yafo, the Karlsruhe Institute of Technology and the Southwest University China and uses CoCoME as demonstrator as well. Nevertheless, the approach has some disadvantages and limitations as depicted in Sec.\ref{sec:stateOfTheArt:comparison}. 
\\
However, one can presume that the proposed microservices in \cite{FunctionalDecompositionHeinrich} are good candidates, as their evaluation included three independent software projects that implemented CoCoME in a similar manner. \\
The following microservices are identified: 
\begin{itemize}
    \item  List of Services
\end{itemize}

\subsection{Reference Set 2: Manual Decomposition}
//Gefahr vs Aufwand reinbringen

\section{Metrics}
\label{sec:Evaluation:Metrics}

\subsection{Precision and Recall}
// Hier noch erklären?

\subsection{Cyclic Dependencies}
//Hier noch erlären

\section{Results}
\label{sec:Evalutation:Results}

\subsection{Identified Microservices}
//Hier veranschaulichen was unser Ansatz gefunden hat


