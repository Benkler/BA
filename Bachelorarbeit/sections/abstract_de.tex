% LaTeX2e class for student theses
%% sections/abstract_de.tex
%% 
%% Karlsruhe Institute of Technology
%% Institute for Program Structures and Data Organization
%% Chair for Software Design and Quality (SDQ)
%%
%% Dr.-Ing. Erik Burger
%% burger@kit.edu
%%
%% Version 1.3.3, 2018-04-17

\Abstract
Angetrieben durch den Aufstieg von Cloud Computing, agilen Entwicklungsmethoden, DevOps und Continuous Deployment Strategien etablierte sich die Microservice Architektur als Alternative zum monolithischen Software Design. Microservices sind eine Ansammlung unabhängiger, in sich zusammenhängende, aber lose gekoppelter Services, die die Defizite zentralisierter, monolithischer Software bewältigen. Namhafte Unternehmen hatten erst kürzlich ihre monolithische Alt-Software in ein microservice-basiertes System überführt. Eine Schlüsselaufgabe dabei ist es, die richtige Aufteilung der Alt-Software zu finden. Dieser Prozess wird Microserviceidentifikation genannt. Bis jetzt wurde er weitestgehend intuitiv und auf Basis von Expertenwissen durchgeführt. Der Hauptgrund dafür liegt vor allem in fehlenden formalen Ansätzen und automatisierter Unterstützung durch Software.\\
Dennoch wachsen Applikation mit der Zeit und werden zunehmend komplexer, sodass die Aufteilung eines Systems in dieser Komplexität durchaus herausfordern ist. Die Thesis stellt daher einen formalen, graph-basierten Ansatz vor, der mittels Clustering-Techniken Microservice-Kandidaten extrahiert. Der Ansatz basiert auf der Prozesssicht und stellt Kontrollfluss- und Datenflussabhängigkeiten als gewichtete Graphen da. Diese werde benutzt, um stark zusammenhängende Aktivitäts- und Datencluster zu identifizieren. Anschließend werden diese Cluster abgeglichen, um zusammenhängende Cluster aus Aktivitäten und zugehörigen Datenobjekten zu erstellen. Jedes Cluster entspricht dann einem Microservice. \\
Eine Evaluierung zeigt, dass der Ansatz vernünftige Microservice-Kandidaten identifiziert, die einer manuellen Zerlegung ähnlich sind, aber mit viel geringerem Zeitaufwand und erforderlichem Fachwissen identifiziert werden.

