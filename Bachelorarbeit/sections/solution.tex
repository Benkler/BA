\chapter{Solution}
\label{ch:Solution}
This chapter presents the solution to tackle the issue of microservice identification. 
As noted in the \textit{State of the Art} chapter \ref{ch:StateOfTheArt}, existing approaches support two initial situations: They either conduct the extraction of microservices from existing (monolithic) systems or they are based on microservice greenfield development. Both types have their advantages and disadvantages. Existing systems, for instance, provide more information about the system specification and requirements. Legacy code and log files can be used to extract data dependencies or process structures. However, shortcoming in the design of the legacy application might have an impact on the extracted information and influence the microservice extraction in a negative manner. In contrast, greenfield development is not affected by any previously committed design decisions. As a matter of fact, the greenfield approach can be applied to existing systems as well by discarding legacy code and additional information that arose during the development. Solely the system requirements that existed before the implementation started serve as input. Nevertheless, this type has to manage the identification process with less input.\\
The solution we propose is based on pre-existing system requirements. No existing implementation is used and consequently, the presented approach is to be classified as greenfield method.\\



\section{Basic Approach}


\vspace{0.5cm}
\par
\begingroup
\leftskip=1cm
\rightskip=1cm

\noindent
\textbf{RQ1: Which is the most appropriate strategy to decompose a system into microservices? }

\endgroup
\vspace{0.5cm}



\noindent
This thesis proposes a formal, graph-based microservice identification approach using clustering on control flow and data flow. It is  inspired by Amiri’s work on \textit{Object-aware Identification of Microservices} \cite{ObjectAwareAmiri}. In chapter \ref{ch:StateOfTheArt}, eight suitable approaches to identify microservices are presented and compared using well-defined criteria. Most of them require special prerequisites and cannot be applied to various types of systems, i.e. no greenfield applications, systems without meaningful VCS meta-data or the absence of log files.
\\
By the way of contrast, Amiri proposes an approach to extract structural and data object dependencies from business point of view in order to generate possible microservice candidates. In doing so, he relinquishes to use any further information besides BPMN models. Using both, structural and data object dependencies promotes high cohesiveness and loose coupling on functional and data object level. In other words, high cohesive functionality is divided into the same microservices, together with the data objects that are accessed. \\
However, Sec.\ref{sec:stateOfTheArt:comparison} outlines the limitations and drawbacks of the approach. Whereas the control flow is depicted clearly, the data flow remains vague. The weight definitions regarding data object dependencies lack formal explanation. Further, the aggregation of structural and data object dependencies, and consequently the aggregation of control flow and data flow contains a significant problem: In Amiri's approach, the aggregation is conducted by summing up two relation matrices. The matrix entries representing the dependencies highly influence the results. For instance, a large amount of data reads and writes sum up to great numbers, outweighing the structural dependencies. Thus, the identification process would be almost based on data dependencies only, discarding any identified structural dependencies. \\
The following sections introduces a formal approach to tackle \textit{RQ2}. First, a basic overview of the approach is provided. Afterwards, each step is introduced in detail including alternatives and examples. 




\vspace{0.5cm}
\par
\begingroup
\leftskip=1cm
\rightskip=1cm

\noindent
\textbf{RQ2: What formal approach can be constructed to identify possible microservices without detailed know-how and manual effort? }

\endgroup
\vspace{0.5cm}



\noindent
To that end, Fig.\ref{fig:thesisProcess} provides an overview of the proposed approach. As depicted previously, the process requires input in form of BPMN models. Therefore, specifying those models marks the beginning of the process. Afterwards, control flow and data flow need to be extracted. To avoid the ambiguity of aggregating data flow and control flow as proposed by Amiri, the approach recommends to create two independent weighted Graphs, using the information from the previous step. In the next step, a clustering algorithm determines two sets of clusters based on the weights in the graphs. At that point, the process determined a set of clusters that is based on the structural dependencies and another one, based on data object dependencies. In the following, a matching process identifies commonalities between data object-based and structural-based clusters in order to create comprehensive clusters. Based on these clusters, the last step extracts microservice candidates. \\
%TODO  stimmt das noch?
\textit{RQ2} also broaches the subject of necessary know-how and the amount of manual effort. As far as that is concerned, the proposed approach does not require human interaction as soon as the BPMN models are specified. Everything beyond that is based on a structural process.
Admittedly, the manual effort to conduct the process entirely is still not to be neglected, as the extraction process, graph creation and cluster matching is not yet automated. Nevertheless, the structural process enables to implement the entire approach, excluding the BPMN model specification step. However, implementing an approach is beyond the scope of the paper.




 
\begin{figure}[h!]
	\includegraphics[width=\textwidth, trim={7.5cm 15.3cm 5.0cm 1.5cm}]{img/ThesisProcess.pdf}
	\caption{Overview of the identification approach}
	\label{fig:thesisProcess}
\end{figure}






\section{Specify BPMN Models}
\label{sec:Solution:SpecifyBPMN}
Chapter \ref{ch:PrepApproach}, introduces the BPMN 2.0 modelling language as an easy to use, but yet powerful notation to illustrate business processes including their activities and their data needs.
In the first step of the solution, those business processes need to be specified. Usually, the system specification are not directly given in form of business processes, bur rather in the form of use cases, UML models, domain models or even as textual description in natural language. Therefore, the first step consists of specifying a business model using the available system specification. This can be achieved using various approaches, for instance: \\

\noindent
\textbf{Workshops} At the very beginning of a software project, technical and non-technical stakeholders can participate in a workshop to specify the business process model. As illustrated in the BPMN specification, the "primary goal of BPMN is to provide a notation that is readily understandable by all business users"  \cite{OMG}. Therefore, carrying out a workshop with stakeholders from various departments can produce high quality BPMN models which can be further used as input for the extraction process.\\

\noindent
\textbf{Use Cases} In the case of CoCoME, the system specifications are available as use cases (cf. chapter \ref{ch:CoCoME}). Accordingly, section \ref{sec:PrepApproach:TransformUCtoBPMN} illustrates a process to transform use cases into BPMN models. 
\\

\noindent
\textbf{Others} Business processes can be extracted in various other ways. UML Activity diagrams, for instance, are very similar to BPMN models. Van der Aalst et al. elaborated the Process Mining Manifesto \cite{ProcessMiningManifesto} where he presents general techniques to extract business processes, although it is mainly event log driven. Another approach is the \textit{BPMN Miner}, an automatic discovery tool for BPMN process models \cite{BPMNMiner}. Again, the tool discovers the models dynamically, using log files of a legacy system.



\section{Extract Control Flow}
\label{sec:Solution:ExtractControlFlow}
In the course of the process, activities of the business processes are clustered based on their structural dependency which is extracted from the control flow. Activities (tasks) in business processes play the role of operations in microservices, representing the functionality a service is able to offer. During the process of microservice identification, it is desired to cluster high cohesive functionality into one microservice. To achieve this, one must first extract the structural  dependencies between activities in business processes. \\
The extraction process itself is trivial, as the business process language BPMN was designed to illustrate the control flow between activities (cf. Sec.\ref{sec:PrepApproach:BPMN}). Mainly inspired by the work of M. Amiri in \textit{Object-aware Identification of Microservices} \cite{ObjectAwareAmiri}, we propose a straightforward technique to separate the control flow information from BPMN 2.0 models in section \ref{ch:PrepApproach:ControlDataFlowBPMNProcess}.




\section{Extract Data Flow}
\label{sec:Solution:ExtractDataFlow}
Besides the structural dependencies of activities, data object access plays a significant role in the definition of microservices. As depicted in the background chapter \ref{ch:background}, microservice generally administer their own database with the data entities that belong to the bounded context of the service. Usually, data needs to be shared among services which raises the question where to place a shared data object. However, sharing data among microservices is expensive because it includes network communication instead of inter process communication. It is therefore desirable to reduce the communication between services by distributing data objects into the same microservice if they are accessed together.  
Like the previous section, we propose to use clustering based on data flow to identify high cohesive but loosely coupled set of data object clusters. \\
When BPMN 2.0 was introduced, the language was extended by the ability to represent data objects that are consumed and/or produced by the activities. Despite the fact that BPMN is still a language to illustrate the control flow of business processes, the extension provides the possibility to visualize the data flow. In Sec.\ref{ch:PrepApproach:ControlDataFlowBPMNProcess}, we present a formal approach to extract the data flow by discarding anything but flow elements used to represent the data flow.



\section{Create a weighted Graph using Control Flow}
\label{sec:Solution:CreateGraphControl}
The visualization of the control flow information as a graph enables to identify clusters of high cohesive functionality. Therefore, we build a directed graph G whose vertices represents the tasks/activities in the business process. The edges correspond to the control flow arcs in the BPMN model: a pair of activities is connected if there is a direct edge in the business process or if there is a path between them that contains only gateways. This decision is based on the heuristic, that two activities are more likely to be in the same microservice, if they are directly connected in a business process. \\
Speaking of the weights, we decide to assign a value of 1 to each edge notwithstanding of the nature of the connection, which is i) directly connected ii) connected via parallel gateway iii) connected via XOR gateway. Regarding the first and the second case, it is motivated by the fact that activities connected by a parallel gateway and activities that are directly connected are always executed during control flow execution.
In regard of the third case, one can argue that the probability of a condition influences the weight of a connection. For instance, a task has two subsequent tasks that are connected through an exclusive OR gateway and conditional flows. One of the tasks, the "main task", is more likely to be the successor as the alternative. Hence, the edges need a different weight. However, the information regarding the probability is usually not available and specified in business processes. Further, different weighting raises the question of the value determination. With this in mind, the generalization of all types of connections (using a weight of 1 for all edges) seems to be an appropriate solution. \\
Fig.\ref{fig:controlFlow} shows an exemplary BPMN process whose object-related information has already been deleted (cf. Sec.\ref{sec:Solution:ExtractControlFlow}). As explained in this section, the control flow information given in this BPMN model is used to create a weighted graph. The corresponding graph is illustrated in Fig.\ref{fig:controlFlowGraph}. 


\begin{figure}[h!]
	\includegraphics[width=\textwidth, trim={4.5cm 14cm 4.0cm 1.5cm}]{img/ControlFlowExample.pdf}
	\caption{An exemplary BPMN model illustrating the control flow only  }
	\label{fig:controlFlow}
\end{figure}
%"l, b, r, t"
\begin{figure}[h!]
	\centering
	\includegraphics[width=12cm, trim={1.5cm 9.0cm 13.0cm 0cm}]{img/ControlFlowGraph.pdf}
	\caption{Weighted graph using Control Flow Dependencies}
	\label{fig:controlFlowGraph}
\end{figure}

 






\section{Create a weighted Graph using Data Flow}
\label{sec:Solution:CreateGraphData}
To identify high cohesive data object clusters, the data flow information is visualized as a weighted graph, which is similar to the activity graph as described in previous sections. In this case, the vertices of a Graph G represent the data objects in the BPMN models. The edges illustrate the data object dependencies extracted from the data flow.
Such dependencies are: i) data objects being accessed by the same task ii) a data object value that is used when writing to (creating) another data object. \\
Speaking of the first dependency, it is obvious that data which is accessed by the same task is more likely to be partitioned into the same service. Otherwise, the execution of a task would always cause at least one expensive intra-service call.
Therefore, two vertices that represent a pair of data objects which is accessed by the same task is to be connected by an edge. \\
The second dependency is based on a similar heuristic. There is a certain connection between two data objects, if information of one data object is used to update or create another one. Placing the information source into another microservice as the information destination would require an inevitable cross-service communication which is meant to be prevented. \\
To determine data flow connections and create a weighted graph, it is necessary to take a closer look at the extracted data flow (cf. Sec.\ref{sec:Solution:ExtractDataFlow}). It is noticeable that it requires additional information to decide whether two 


\section{Identify Cluster}
\label{sec:Solution:IdentifyCluster}

\section{Match Cluster}
\label{sec:Solution:MatchCluster}
Having both sets of clusters...

\section{Extract Microservice Candidates}
\label{sec:Solution:ExtractMicroserviceCandidates}




\section{Distance Metric}
\begin{itemize}
	\item Determine distance between pair of Objects in BPMN (Count activities in between)
	\item Basic Idea: Data accessed more successively is more connected and belongs to same service
	\item Pro: Easy
	\item: Contra: Not really Data Flow; pair  occurs several times --> On time close, other times far away, what about average?
\end{itemize}

\section{Count Activities Metric}
\begin{itemize}
	\item Count activities that access a pair of objects in BPMN
	\item Basic Idea: Data frequently accessed together is more connected and belongs to the same service
	\item Pro: represents idea of data cohesiveness
	\item Contra: Again activities (no clean data flow) but nothing else!

\end{itemize}





